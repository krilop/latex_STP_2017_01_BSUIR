\section{Минимизация логической функции}

\subsection{Постановка задачи}

Минимизировать функцию алгебры логики(ФАЛ), представленную в
дизъюнктивной нормальной форме и содержащую конъюнктивные термы ,
методом неопределенных коэффициентов.

\begin{equation}
\label{equation:section2:F}
     f(a,b,c) = {\overline{a}b\overline{c}} \lor {\overline{a}\overline{b}\overline{c}} \lor {ab\overline{c}} \lor {a\overline{b}c}
\end{equation}
\subsection{Описание алгоритма минимизации логической функции}

Описываемый здесь метод может быть применен для минимизации
ФАЛ от любого числа аргументов, однако для простоты изложения и большей наглядности его рассмотрение будем производить на примере минимизации функции, зависящей от трех аргументов. Представим функцию $f(x_1,x_2,x_3)$ в следующем виде:

\begin{equation}
\label{equation:sec2:viewF}
\begin{split}
f(x_1,x_2,x_3) = & K_1^1x_1 \lor K_1^0\overline{x_1} \lor K_2^1x_2 \lor K_2^0\overline{x_2} \lor K_3^1x_3 \lor K_3^0\overline{x_3} \\
&\lor K_{12}^{11}x_1x_2 \lor K_{12}^{10}x_1\overline{x_2} \lor K_{12}^{01}\overline{x_1}x_2 \lor K_{12}^{00}\overline{x_1}\overline{x_2} \\
&\lor K_{13}^{11}x_1x_3 \lor K_{13}^{10}x_1\overline{x_3} \lor K_{13}^{01}\overline{x_1}x_3 \lor K_{13}^{00}\overline{x_1}\overline{x_3} \\
&\lor \overline{x_3} \lor K_{23}^{11}x_2x_3 \lor K_{23}^{10}x_2\overline{x_3} \lor K_{23}^{01}\overline{x_2}x_3 \lor K_{23}^{00}\overline{x_2}\overline{x_3} \\
&\lor K_{123}^{111}x_1x_2x_3 \lor K_{123}^{110}x_1x_2\overline{x_3} \lor K_{123}^{101}x_1\overline{x_2}x_3 \\
&\lor K_{123}^{100}x_1\overline{x_2}\overline{x_3} \lor K_{123}^{011}\overline{x_1}x_2x_3 \lor K_{123}^{010}\overline{x_1}x_2\overline{x_3} \\
&\lor K_{123}^{001}\overline{x_1}\overline{x_2}x_3 \lor K_{123}^{000}\overline{x_1}\overline{x_2}\overline{x_3}
\end{split}
\end{equation}


Здесь представлены все возможные конъюнктивные члены, которые могут входить в дизъюнктивную форму представления функции. Коэффициенты $K$ с различными индексами являются неопределенными и подбираются так, чтобы получающаяся после этого дизъюнктивная форма была минимальной. Если теперь задавать всевозможные наборы значений аргументов и приравнивать полученное после этого выражение (отбрасывая нулевые конъюнкции) значению функции на выбранных наборах, то получим систему $2^3$ уравнений для определения коэффициентов $K$. Получаем следующую систему:

\[
\begin{cases}
K_1^0 \lor K_2^0 \lor K_3^0 \lor K_{12}^{00} \lor K_{13}^{00} \lor K_{23}^{00} \lor K_{123}^{000}&=f(0,0,0),\\
K_1^0 \lor K_2^0 \lor K_3^1 \lor K_{12}^{00} \lor K_{13}^{01} \lor K_{23}^{01} \lor K_{123}^{001}&=f(0,0,1),\\
K_1^0 \lor K_2^1 \lor K_3^0 \lor K_{12}^{01} \lor K_{13}^{00} \lor K_{23}^{10} \lor K_{123}^{010}&=f(0,1,0),\\
K_1^0 \lor K_2^1 \lor K_3^1 \lor K_{12}^{01} \lor K_{13}^{01} \lor K_{23}^{11} \lor K_{123}^{011}&=f(0,1,1),\\
K_1^1 \lor K_2^0 \lor K_3^0 \lor K_{12}^{10} \lor K_{13}^{10} \lor K_{23}^{00} \lor K_{123}^{100}&=f(1,0,0),\\
K_1^1 \lor K_2^0 \lor K_3^1 \lor K_{12}^{10} \lor K_{13}^{11} \lor K_{23}^{01} \lor K_{123}^{101}&=f(1,0,1),\\
K_1^1 \lor K_2^1 \lor K_3^0 \lor K_{12}^{11} \lor K_{13}^{10} \lor K_{23}^{10} \lor K_{123}^{110}&=f(1,1,0),\\
K_1^1 \lor K_2^1 \lor K_3^1 \lor K_{12}^{11} \lor K_{13}^{11} \lor K_{23}^{11} \lor K_{123}^{111}&=f(1,1,1).
\end{cases}
\]

Пусть таблично задана некоторая функция. Задание некоторой конкретной функции определяет значения правых частей системы. Если набор таков, что функция на этом наборе равна нулю, то в правой части соответствующего уравнения будет стоять нуль.
Для удовлетворения этого уравнения необходимо приравнять нулю все коэффициенты $K$, входящие в левую часть рассматриваемого уравнения. (Это вытекает из того, что дизъюнкция равна нулю только тогда, когда все члены, входящие в нее, равны нулю).

Рассмотрев все наборы, на которых данная функция обращается в нуль, получим все нулевые коэффициенты $К$. В уравнениях, в которых справа стоят единицы, вычеркнем слева все нулевые коэффициенты. Из оставшихся коэффициентов приравняем единице коэффициент, определяющий конъюнкцию наименьшего возможного ранга, а остальные коэффициенты в левой части данного уравнения примем равными нулю (это можно сделать, так как дизъюнкция обращается в единицу, если хотя бы один член ее равен единице). Единичные коэффициенты определят из (\ref{equation:sec2:viewF}) соответствующую дизъюнктивную нормальную форму. 
\cite{koeff}

\subsection{Минимизация логической функции}
Составим таблицу истинности для функции (\ref{equation:section2:F}):
\begin{table}[ht]
\caption{Таблица истинности функции (\ref{equation:section2:F})}
\centering
\begin{tabularx}{\linewidth}{|C|C|C|C|}
\hline
a & b & c & f \\ \hline
0 & 0 & 0 & 1 \\ \hline
0 & 0 & 1 & 0 \\ \hline
0 & 1 & 0 & 1 \\ \hline
0 & 1 & 1 & 0 \\ \hline
1 & 0 & 0 & 0 \\ \hline
1 & 0 & 1 & 1 \\ \hline
1 & 1 & 0 & 1 \\ \hline
1 & 1 & 1 & 0 \\ \hline
\end{tabularx}
\end{table}

Следуя алгоритму, составим систему логических уравнений из коэффициентов при всех возможных конъюнктивных членах, используя таблицу получненную шагом ранее:\\
\[
\begin{cases}
K_1^0 \lor K_2^0 \lor K_3^0 \lor K_{12}^{00} \lor K_{13}^{00} \lor K_{23}^{00} \lor K_{123}^{000}&=1,\\
K_1^0 \lor K_2^0 \lor K_3^1 \lor K_{12}^{00} \lor K_{13}^{01} \lor K_{23}^{01} \lor K_{123}^{001}&=0,\\
K_1^0 \lor K_2^1 \lor K_3^0 \lor K_{12}^{01} \lor K_{13}^{00} \lor K_{23}^{10} \lor K_{123}^{010}&=1,\\
K_1^0 \lor K_2^1 \lor K_3^1 \lor K_{12}^{01} \lor K_{13}^{01} \lor K_{23}^{11} \lor K_{123}^{011}&=0,\\
K_1^1 \lor K_2^0 \lor K_3^0 \lor K_{12}^{10} \lor K_{13}^{10} \lor K_{23}^{00} \lor K_{123}^{100}&=0,\\
K_1^1 \lor K_2^0 \lor K_3^1 \lor K_{12}^{10} \lor K_{13}^{11} \lor K_{23}^{01} \lor K_{123}^{101}&=1,\\
K_1^1 \lor K_2^1 \lor K_3^0 \lor K_{12}^{11} \lor K_{13}^{10} \lor K_{23}^{10} \lor K_{123}^{110}&=1,\\
K_1^1 \lor K_2^1 \lor K_3^1 \lor K_{12}^{11} \lor K_{13}^{11} \lor K_{23}^{11} \lor K_{123}^{111}&=0.
\end{cases}
\]

Далее уберем из системы коэффициенты заведомо равные нулю:\\
\[
\begin{cases}
\cancel{K_1^0} \lor \cancel{K_2^0} \lor \cancel{K_3^0} \lor \cancel{K_{12}^{00}} \lor K_{13}^{00} \lor \cancel{K_{23}^{00}} \lor K_{123}^{000}&=1,\\
\cancel{K_1^0} \lor \cancel{K_2^0} \lor \cancel{K_3^1} \lor \cancel{K_{12}^{00}} \lor \cancel{K_{13}^{01}} \lor \cancel{K_{23}^{01}} \lor \cancel{K_{123}^{001}}&=0,\\
\cancel{K_1^0} \lor \cancel{K_2^1} \lor \cancel{K_3^0} \lor \cancel{K_{12}^{01}} \lor K_{13}^{00} \lor K_{23}^{10} \lor K_{123}^{010}&=1,\\
\cancel{K_1^0} \lor \cancel{K_2^1} \lor \cancel{K_3^1} \lor \cancel{K_{12}^{01}} \lor \cancel{K_{13}^{01}} \lor \cancel{K_{23}^{11}} \lor \cancel{K_{123}^{011}}&=0,\\
\cancel{K_1^1} \lor \cancel{K_2^0} \lor \cancel{K_3^0} \lor \cancel{K_{12}^{10}} \lor \cancel{K_{13}^{10}} \lor \cancel{K_{23}^{00}} \lor \cancel{K_{123}^{100}}&=0,\\
\cancel{K_1^1} \lor \cancel{K_2^0} \lor \cancel{K_3^1} \lor \cancel{K_{12}^{10}} \lor \cancel{K_{13}^{11}} \lor \cancel{K_{23}^{01}} \lor K_{123}^{101}&=1,\\
\cancel{K_1^1} \lor \cancel{K_2^1} \lor \cancel{K_3^0} \lor \cancel{K_{12}^{11}} \lor \cancel{K_{13}^{10}} \lor K_{23}^{10} \lor K_{123}^{110}&=1,\\
\cancel{K_1^1} \lor \cancel{K_2^1} \lor \cancel{K_3^1} \lor \cancel{K_{12}^{11}} \lor \cancel{K_{13}^{11}} \lor \cancel{K_{23}^{11}} \lor \cancel{K_{123}^{111}}&=0.
\end{cases}
\]

И получим следующую систему:\\
\[
\begin{cases}
 K_{13}^{00} \lor K_{123}^{000}&=1,\\
 K_{13}^{00} \lor K_{23}^{10} \lor K_{123}^{010}&=1,\\
 K_{123}^{101}&=1,\\
 K_{23}^{10} \lor K_{123}^{110}&=1,\\
\end{cases}
\] 

Отсюда находим минимальную дизъюнктивную нормальую форму данной функции:

\begin{equation*}
f(a,b,c)={\overline{a}*\overline{c}} \lor {b*\overline{c}} \lor {a*\overline{b}*c}
\end{equation*}

Сравним эту данную форму функции с результатом полученным путем минимизации методом карты Карно. На рисунке \ref{fig:section2:karno} представлен результат минимизации функции \ref{equation:section2:F}:

\begin{figure}[ht!]
    \centering
    \includegraphics[scale=1.1]{images/S2IM1.png}
    \caption{Форма функции полученная при помощи метода минимизацией картами Карно}
    \label{fig:section2:karno}
\end{figure}
\vspace{-1.5em}
\subsection{Выводы}
В результате выполнения задания по минимизации логической функции
используя метод неопределенных коэффициэнтов мы нашли эквивалентное выражение
заданной функции алгебры логики в форме, содержащей минимально
возможное число переменных. Результат минимизации совпал с формой полученной путем минимизации картой Карно.
