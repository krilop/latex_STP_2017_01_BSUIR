\section{Понятие абстрактного цифрового автомата}

\subsection{Постановка задачи}

%Привести обобщенную структурную схему автомата Мили. Пояснить векторное представление абстрактного автомата: описание входов/выходов, начального состояния, функций переходов и функций выходов, состояний памяти. Пояснить способ табличного представления работы автомата и в виде графа автомата.


Привести обобщенную структурную схему автомата Мили. Пояснить векторное представление абстрактного автомата: описание входов/выходов, начального состояния, функций переходов и функций выходов, состояний памяти. Пояснить способ табличного представления работы автомата и в виде графа автомата.

\subsection{Обобщенная структурная схема автомата Мили}

В отличие от абстрактного автомата структурный цифровой автомат имеет $L$ входов и $N$ выходов. На входы структурного автомата поступают наборы входных двоичных переменных из множества $X = \lbrace x_1,x_2,...,x_L \rbrace$, а на выходах формируются выходные двоичные сигналы из множества $Y= \lbrace y_1,y_2,...,y_N \rbrace$ . Структурная модель автомата представляет собой две взаимосвязанные части: комбинационную схему и память. Комбинационная часть автомата кроме сигналов из множества $Y$ формирует также двоичные сигналы, подаваемые на входы элементов памяти $D=\lbrace d_1,d_2,...,d_r \rbrace$. Эти сигналы называются функциями возбуждения элементов памяти и представляют собой код состояния перехода. Сигналы, формируемые на выходах элементов памяти $T =\lbrace \tau_1, \tau_2, …, \tau_r\rbrace$, подаются на входы комбинационной схемы наряду с входными переменными и называются переменными обратной связи. Переменные обратной связи являются кодом текущего состояния автомата. На рисунке \ref{fig:section4:generalizedSchemeDifitalMachine} представленна обобщенная структурная схема цифрового автомата:

\begin{figure}[ht!]
    \centering
    \includegraphics[scale=0.8]{S4IM1.png}
    \caption{Обобщенная структурная схема цифрового автомата}
    \label{fig:section4:generalizedSchemeDifitalMachine}
\end{figure}

\subsection{Векторное представление абстрактного автомата}

Абстрактный автомат применительно к теории и практике ЭВС – это математическая модель цифрового автомата, задаваемая шестикомпонентным вектором $S=(A,Z,W,\delta,\lambda,a_1)$, где $A=\lbrace a_1 ,..., a_m \rbrace$ – множество внутренних состояний абстрактного автомата; $Z=\lbrace z_1 ,..., z_m \rbrace$ и $W =\lbrace w_1 ,... ,w_m \rbrace$ – соответственно множества входных и выходных абстрактных слов; $\delta (Z, A)$ - функция переходов от входного слова к последующему внутреннему состоянию с учетом предыдущего; $\lambda (A, W)$ - функция выходов, устанавливающая зависимость выходных абстрактных слов от внутренних состояний автомата; $a_1$ – начальное внутреннее состояние автомата.

\subsection{Способы представления работы автомата (табличный, в виде графа)}

Закон функционирования автоматов может быть задан в виде систем уравнений, таблиц, матриц и графов. В рамках курсовой работы будет описаны только задания с помощью таблицы и графа. Под законом функционирования понимается совокупность правил, описывающих переходы автомата в новое состояние и формирование выходных символов в соответствии с последовательностью входных символов. В зависимости от типа автомата при табличном задании закона функционирования автомата используются либо таблицы переходов и выходов (автомат Мили), либо совмещенная таблица переходов и выходов (автомат Мура).

\subsubsection{}
Задание автомата с помощью таблицы. 
В зависимости от типа автомата при табличном задании закона функционирования автомата используются либо таблицы переходов и выходов (автомат Мили), либо совмещенная таблица переходов и выходов (автомат Мура). С помощью таблиц \ref{table:section4:jump} и \ref{table:section4:output} (таблицы переходов и таблицы выходов соответственно), задан закон функционирования абстрактного автомата Мили, для которого $A=\lbrace a_1,a_2,a_3,a_4\rbrace$, $Z=\lbrace z_1,z_2,z_3\rbrace$,  $W=\lbrace w_1,w_2,w_3,w_4,w_5\rbrace$, где $A=\lbrace a_1 ,..., a_m \rbrace$ – множество внутренних состояний абстрактного автомата; $Z=\lbrace z_1 ,..., z_m\rbrace$ и $W =\lbrace w_1 ,... ,w_m \rbrace$ – соответственно множества входных и выходных абстрактных слов.
%ТАБЛИЦЫ СЮДА ПИХНУТЬ НАДО

\begin{table}[ht]
\caption{Таблица переходов автомата Мили}
\label{table:section4:jump}
\centering
\begin{tabularx}{\linewidth}{|C|C|C|C|C|}
\hline
$\delta$ & $a_1$ & $a_2$ & $a_3$ & $a_4$ \\ \hline
$z_1$    & $a_2$ & $a_2$ & -     & $a_4$ \\ \hline
$z_2$    & $a_4$ & -     & $a_1$ & $a_3$ \\ \hline
$z_3$    & $a_3$ & $a_3$ & $a_4$ & -     \\ \hline
\end{tabularx}
\end{table}

\begin{table}[ht]
\caption{Таблица выходов автомата Мили}
\label{table:section4:output}
\centering
\begin{tabularx}{\linewidth}{|C|C|C|C|C|}
\hline
$\lambda$ & $a_1$ & $a_2$ & $a_3$ & $a_4$ \\ \hline
$z_1$     & -     & $w_2$ & -     & $w_3$ \\ \hline
$z_2$     & $w_2$ & -     & $w_4$ & $w_5$ \\ \hline
$z_3$     & $w_3$ & $w_1$ & $w_3$ & -     \\ \hline
\end{tabularx}
\end{table}

Входные символы и состояния, которыми помечены строки и столбцы, относятся к моменту времени $t$. В таблице \ref{table:section4:jump} (таблице переходов) на пересечении строки $z_i(t)$ и столбца $a_m(t)$ ставится состояние $a_s (t+1) = \delta ( a_m(t), z_i(t) )$. В таблице \ref{table:section4:output} (таблице выходов) на пересечении строки $z_i(t)$ и столбца $a_m(t)$ ставится выходной символ $w(t) = \lambda ( a_m(t), z_i(t) )$, соответствующий переходу из состояния аm в состояние $a_s$. Таким образом, по таблицам переходов и выходов можно проследить последовательность работы автомата. Так, например, в начальный момент времени $t=0$ автомат, находясь в состоянии $a_1$ (первый столбец), под действием входного символа $z_1$ может перейти в состояние $a_2$, при этом выходной символ не формируется. Под действием входного символа $z_2$ – может перейти в состояние $a_4$ с формированием выходного символа $w_2$, а под действием символа $z_3$ - в состояние $a_3$ с формированием выходного символа $w_3$. 
Далее если на вход автомата, установленного в исходное состояние $а_m \subseteq A$, в моменты времени $t =\lbrace 1,2,…, n\rbrace$ подается некоторая последовательность букв входного алфавита (входных символов) $z_i \subseteq Z$, то на выходе автомата будут последовательно формироваться буквы выходного алфавита (выходные символы) $w_j \subseteq W$, при этом автомат будет переключаться в состояния $a_s \subseteq A$. Следовательно, с помощью таблиц переходов и выходов можно получить выходную реакцию автомата на любое входное слово.

\subsubsection{}
Задание автомата в виде графа.
Граф автомата – ориентированный граф, вершины которого соответствуют состояниям, а дуги − переходам между ними. Дуга, направленная из вершины $a_m$ в вершину $a_s$, соответствует переходу из состояния $a_m$ в $a_s$. В начале дуги записывается входной символ $z_i$, влияющий на переход $a_s = \delta(a_m, z_i)$, а символ $w_j$ записывается в конце дуги (автомат Мили) или рядом с вершиной (автомат Мура).\cite{demid}  На рисунке \ref{fig:section4:grafOfMili} представлен пример задания автомата Мура в виде графа:

\begin{figure}[ht!]
    \centering
    \includegraphics[scale=0.5]{S4IM2.png}
    \caption{представление автомата Мили в виде графа}
    \label{fig:section4:grafOfMili}
\end{figure}


