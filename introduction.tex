\csection{Введение}
Курсовое проектирование является обязательным компонентом обучения студентов в высшем учебном заведении и одной из форм текущей оценки студента по учебной дисциплине. Это первый проект такого масштаба для студентов, который содержит результаты теоретических и экспериментальных исследований по дисциплине "Арифметические и логические основы цифровых устройств". Курсовая работа включает в себя аналитические, расчетные, экспериментальные и конструкторские задания, а также разработку графической документации, что помогает студентам приобретать навыки разработки и проектирования цифровой электроники.

Первый раздел посвящен вычитанию с чисел с плавающей запятой, а второй - минимизации ФАЛ методом неопределенных коэффициентов. Оба раздела являются частью первого учебного семестра дисциплины "Арифметические и логические основы цифровых устройств" и содержат материалы, посвященные двоичной арифметике и логическим основам цифровых устройств.

Третий раздел посвящен принципу разработки структурной схемы цифрового устройства "Мультиплексор". В этом разделе была сформирована таблица истинности для шифратора и создана структурная схема.

Четвертый и пятый разделы содержат материал по теме "Цифровые автоматы" из второго семестра. Раздел 4 содержит теоретические сведения о цифровых автоматах Мура, векторном представлении абстрактного автомата и способах его представления. Раздел 5 посвящен синтезу цифрового автомата Мура.