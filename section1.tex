\section{Двоичная арифметика}

\subsection{Поcтановка задачи}

Представить алгоритм и выполнить вычитание чисел 95,125 - 11 в двоичной форме с плавающей запятой, заданными в десятичной форме.

\subsection{Описание алгоритма вычитания}

При вычитании чисел с плавающей запятой используется сложение в дополнительном коде, а складываемые цифры (разряды) должны иметь одинаковый вес. Это требование выполняется, если вычитаемые числа
имеют одинаковые порядки. Если число положительное, то дополнительный(доп) код этого числа такой же, как и прямой(пр). Если число отрицательное, то доп. код числа можно получить путем инвертирования каждого разряда числа, кроме знакового, и добавлением к младшему разряду единицу. Алгоритм перевода из доп. кода в пр. код такой же. Пусть имеются два числа $A$ и $B$ с плавающей запятой:

$A = \pm m_A r^{\pm p_A}$, 

$B = \pm m_B r^{\pm p_B}$,\begin{explanation}
    \item[где] $m$ - мантисса числа;
    \item $r$ - основание системы счисления(в нашем случае $r$ = 2);
    \item $p$ - порядок числа. 
\end{explanation}

Алгоритм вычитания чисел с произвольными знаками состоит
в следующем.
\begin{enumerate}
    \item Произвести сравнение порядков $p_A$ и $p_B$. Для этого из порядка числа $A$
вычитается порядок числа $B$. Разность $p = p_A − p_B$ указывает, на сколько раз-
рядов требуется сдвинуть вправо мантиссу числа с меньшим порядком. Если
$p = p_A − p_B > 0$, то $p_A > p_B$ и для выравнивания порядков необходимо сдвинуть
вправо мантиссу $M_B$. Если $p = p_A − p_B < 0$, то $p_B > p_A$, и для выравнивания поряд-
ков необходимо сдвинуть вправо мантиссу $M_A$. Если $p = p_A − p_B = 0$, то $p_A = p_B$, и
порядки слагаемых выравнивать не требуется.
    \item Выполнить сдвиг соответствующей мантиссы на один разряд, повторяя
его до тех пор, пока $p ≠ 0$.
    \item Перевести числа из пр. кода в доп. код.
    \item Выполнить сложение мантисс $M_A$ и $M_B$ по правилу сложения правильных дробей.
    \item Если при сложении мантисс произошло переполнение, то
получившийся разряд переполнения игнорировать. Перевести результат сложения из доп. кода в пр. код.
    \item Конец алгоритма.\cite{lucik}
\end{enumerate}

\subsection{Перевод чисел в двоичную систему счисления}

Для перевода \textit{целой} части чисел в двоичную систему счисления воспользуемся
методом деления на основание системы счисления. Последовательно
выполняя операцию деления на основание системы счисления и записывая
остатки от деления в обратном порядке, получим двоичное представление
числа.


На рисунке \ref{fig:section1:conversionOfInteger} представлены переводы целой части чисел из десятичной системы счисления в двоичную:
\begin{figure}[ht!]
    \centering
    \includegraphics[scale=1]{S1IM1.png}
    \caption{Перевод целой части числа}
    \label{fig:section1:conversionOfInteger}
\end{figure}

Для перевода \textit{дробной} части чисел в двоичную систему счисления воспользуемся
методом умножения на основание системы счисления. Последовательно
выполняя операцию умножения на основание системы счисления и записывая целую часть от полученного числа (при получении в целой части единицы, после её записи продолжать алгоритм, считая, что целая часть теперь равна нулю), получим двоичное представление дробной части числа.

На рисунке \ref{fig:section1:conversionOfFloat} представлен перевод дробной части числа $A$ из десятичной системы счисления в двоичную.

\begin{figure}[ht!]
    \centering
    \includegraphics[scale=1]{S1IM2.png}
    \caption{Перевод дробной части числа $A$}
    \label{fig:section1:conversionOfFloat}
\end{figure}

\subsection{Вычитание двоичных чисел}

% Подход, основанный на решении задачи за счет обучения на основе некоторого набора прецедентов,
% создает ряд существенных преимуществ методов машинного обучения по сравнению со
% многими альтернативными методами, такими как ручной анализ, жестко запрограммированные правила
% и простые статистические модели.
  $A_{10} = +95.125 = A_2 = 0.1011111,001$, 
  $B_{10} = -11 = B_2 = 1.011$,
  $A_{ \text{н}} = 0.1011111001 * 2^7$,   $B_{\text{н}} = 0.1011 * 2^4$,
  $p_A = 7$,    $p_B = 4$

    
Произведем вычитание полученных чисел используя описанный алгоритм:
\begin{enumerate}
 
      \item Сравнение порядков чисел: $\triangle p = p_A - p_B = 7_{10}-4_{10}=3_{10}=0.11_2$

      \item $\triangle p = 3$, следовательно $p_A > p_B$ и для их выравнивания необходимо сдвинуть вправо мантиссу числа $B$ на 3 разряда.

      \item Мантисса числа $B$ после выравнивания и с учетом перевода в дополнительный код(Здесь и далее): $m_B = 0,0001011$, $[m_B]_{\text{доп}}=1,1110101$.

      \item Cумма мантисс:\\
      $[m_A]_{\text{доп}} +[m_B]_{\text{доп}} = [m_C]_{\text{доп}} = 0,1011111001+1,1110101=\cancel{1}0,1010100001$\\
      $[m_C]_{\text{доп}}=0,1010100001=[m_C]_{\text{пр}}$

      \item C учетом порядков сумма чисел:\\
      $C = A+(-B)=0,1010100001*2^7=0.1010100,001_2=84,125_{10}$
      
\end{enumerate}

\subsection{Выводы}

В результате выполнения задания по вычитанию чисел с
плавающей запятой в двоичной форме, заданными в десятичной форме,
был представлен алгоритм вычитания. Результат вычитания чисел в двоичной
форме совпал с результатом вычитания этих же чисел в десятичной форме.