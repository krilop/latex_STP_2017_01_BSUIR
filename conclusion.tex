\csection{Заключение}

Во время выполнения курсовой работы были предствавлены и
применены алгоритмы выполнения операции вычитания двоичных чисел с
плавающей запятой, перевода числа из прямого в дополнительный код и обратно. Применён метод минимизации логических функций с помощью методов неопределенных коэффициентов, карт Карно, с применением закона склеивания для последующего синтеза и проектирования комбинационных и структурных схем цифровых устройств. На основании полученных навыков был выполнен синтез комбинационной схемы демультиплексора. Была изучена теория по теме: «Абстрактные цифровые автоматы», а также разработана граф-схема работы и синтезирована
структура синхронного цифрового автомата Мура. Поставленные задачи были вополнены в полном объеме.