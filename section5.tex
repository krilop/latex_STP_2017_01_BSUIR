\section{Синтез структуры синхронного автомата Мура}

\subsection{Постановка задачи}
Синтезировать структурную схему синхронного цифрового автомата Мура с сигналом на входе в виде меандровой последовательности, 4-я возможными состояниями памяти и двумя выходами, включающими: присутствие на первом выходе автомата сигнала состояния памяти на первом и третьем шаге (интервале времени), а на втором выходе – состояния памяти на втором и четвертом шаге (интервале времени) работы автомата в цикле. 
% ТУТ ИСПРАВИТЬ ТАБЛИЦУ. ВООБЩЕ ПОМЕНЯТЬ
\begin{table}[ht]
\caption{Таблица состояний памяти автомата}
\label{table:section5:staitment}
\centering
\begin{tabularx}{\linewidth}{|C|C|C|C|}
\hline
\multicolumn{4}{|c|}{Состояние автомата в каждом очередном такте работы автомата} \\ \hline
0 0 & 1 0 & 0 1 & 1 1 \\ \hline
\end{tabularx}
\end{table}





\subsection{Построение графа работы автомата Мура}
Для синтеза автомата Мура изобразим граф-схему, соответствующую техническому заданию. Представим её на рисунке \ref{fig:section5:grafOfMura}:

\begin{figure}[ht!]
    \centering
    \includegraphics[scale=0.08]{S5IM1.png}
    \caption{представление автомата Мура в виде графа}
    \label{fig:section5:grafOfMura}
\end{figure}
В качестве элемента памяти выберем простейший T-триггер. Для
определения количества элементов памяти воспользуемся формулой\cite{Iskra}:
\begin{equation*}
     K=\lceil\log _2A\rceil=\log _24=2.
\end{equation*}
\subsection{Построение таблицы переходов автомата Мура}
По результатам изображения граф-схемы работы автомата, построим
таблцу переходов \ref{table:section5:transitionTable}. В ней обозначим текущее состояние памяти автомата за
$a_{1,2}(t)$ , текущее состояние за $a_{1,2}(t+1)$ , вход автомата - $Z$ , функции триггеров -
$T_1$, $T_2$.

  \begin{table}[ht]
\caption{Таблица состояний памяти автомата}
\label{table:section5:transitionTable}
\centering
\begin{tabularx}{\linewidth}{|C|C|C|C|C|C|C|}
\hline
$a_1(t)$ & $a_2(t)$ & $a_1(t+1)$ & $a_2(t+1)$ & $Z$ & $T_1$ & $T_2$ \\ \hline
0        & 0        & 0          & 0          & 0   & 0     & 0     \\ \hline
0        & 0        & 1          & 0          & 1   & 1     & 0     \\ \hline
1        & 0        & 1          & 0          & 0   & 0     & 0     \\ \hline
1        & 0        & 0          & 1          & 1   & 1     & 1     \\ \hline
0        & 1        & 0          & 1          & 0   & 0     & 0     \\ \hline
0        & 1        & 1          & 1          & 1   & 1     & 0     \\ \hline
1        & 1        & 1          & 1          & 0   & 0     & 0     \\ \hline
1        & 1        & 0          & 0          & 1   & 1     & 1     \\ \hline
\end{tabularx}
\end{table}

Исходя из задания понимаем, что после получения нового состояния автомата, это состояние нужно подать на вход комбинационной схемы(КС).
\subsection{Синтез комбинационной схемы функций выходов}
Согласно условию, на первом и третьем шаге автомат должен передавать состояние автомата на первый выход, а на втором и четвертом - на второй выход. Заметим, что на \textit{первый} выход нужно передавать на \textit{нечетных} шагах, а на \textit{второй} выход - \textit{на четных}. Тогда добавим в КС D-триггер с начальным состоянием ноль(Будем называть его счетчик). Каждую итерацию работы автомата будем инвертировать внутреннее состояние этого триггера ($D(t+1)=\overline{D(t)}$), тогда будем получать поочередно ноль или единицу. Будем передавать значение этого счетчика на адресный вход демультиплексора. На информационный вход будем подавать два бита состояния автомата. У данного мультиплексора будет два двухбитовых выхода $Y_1$, $Y_2$. Таблица истинности для функций выхода автомата \ref{table:section5:outputTable}:
  \begin{table}[ht]
\caption{Таблица функций выхода автомата}
\label{table:section5:outputTable}
\centering
\begin{tabularx}{\linewidth}{|C|C|C|C|C|}
\hline
$a_1$ & $a_2$ & $D$ & $y_1$ & $y_2$ \\ \hline
0     & 0     & 0   & 00    & --     \\ \hline
0     & 0     & 1   & --    & 00     \\ \hline
0     & 1     & 0   & 01    & --     \\ \hline
0     & 1     & 1   & --    & 01     \\ \hline
1     & 0     & 0   & 10    & --     \\ \hline
1     & 0     & 1   & --    & 10     \\ \hline
1     & 1     & 0   & 11    & --     \\ \hline
1     & 1     & 1   & --    & 11     \\ \hline
\end{tabularx}
\end{table}
\subsection{Минимизация полученных функций}
Для начала минимизируем функции возбужнения Т-триггеров состояния памяти автомата:

\begin{equation*}
    T_1 = z\overline{a_1}\overline{a_2} \lor za_1\overline{a_2} \lor z\overline{a_1}a_2 \lor za_1a_2 = z;\\    
\end{equation*}
\begin{equation*}
    T_2 = za_1\overline{a_2} \lor za_1a_2 = za_1.    
\end{equation*}
Что касается КС функций выхода, будем считать, что она минимизации не подлежит, ибо перед тем, как войти в демультиплексор, сигналы не проходят через какие-то сложные функции.
\subsection{Построение автомата}
Перед началом построения автомата заметим, что данная схема нуждается в синхронизации. Добавим в качестве входа в автомат генератор тактов. Также добавим два D-триггера, которые будут принимать значение из D-триггеров, чтобы исключить ошибки в работе автомата, связанные с разной скоростью работы его отдельных компонентов. Будем подавать на входы синхронизации T-триггеров значение тактового генератора, а на входы синхронизации D-триггеров задержки состояния автомата и счетчика - его инверсию. В результате построения автомата была получена схема(см. приложение А)


Проведем симуляцию работы получившегося синхронного автомата в программном обеспечении «\textit{logisim-evoltion}». Результат симуляции представлен на рисунке \ref{fig:section5:simulation}:

\begin{figure}[H]
    \centering
    \includegraphics[scale=0.5]{S5IM3.png}
    \caption{Симуляция работы цифрового автомата}
    \label{fig:section5:simulation}
\end{figure}

\subsection{Выводы}
В результате синтеза структуры синхронного автомата Мура была
разработана граф-схема работы автомата Мура, построена таблица переходов
для данного автомата и получены минимальные формы функций возбуждения
триггеров. Также основываясь на принципах синхронизации была скорректирована работа автомата.