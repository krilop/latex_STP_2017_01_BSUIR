\section{Синтез комбинационной схемы}

\subsection{Постановка задачи}

Синтезировать демультиплексор, на вход которого синхронно во времени с трехразрядным кодом адреса поступает информационный сигнал, который следует передать на один из пяти выходов 1, 2, 3, 4, 5, соответствующий коду адреса на входе демультиплексора. Задать в табличной форме функцию переходов и синтезировать комбинационную схему демультиплексора с одним информационными входом, трех разрядным кодом адреса и пятью выходами.

\subsection{Алгоритм синтеза комбинационной схемы}

\begin{enumerate}
    \item Составление таблицы истинности.
    \item Запись функций переходов.
    \item Построение комбинационной схемы.
\end{enumerate}

\subsection{Получение функции выходов}

Для нахождения совершенных дизъюнктивных нормальных форм(СДНФ) составим таблицу истинности цифрового
устройства:


\begin{table}[ht]
\caption{Таблица истинности демультиплексора}
\centering
\begin{tabularx}{\linewidth}{|C|C|C|C|C|C|C|C|}
\hline
$A_1$ & $A_2$ & $A_3$ & $Q_0$ & $Q_1$ & $Q_2$ & $Q_3$ & $Q_4$ \\ \hline
0  & 0   & 0  & $D$   & 0    & 0    & 0    & 0    \\ \hline
0  & 0   & 1  & 0     & $D$  & 0    & 0    & 0    \\ \hline
0  & 1   & 0  & 0     & 0    & $D$  & 0    & 0    \\ \hline
0  & 1   & 1  & 0     & 0    & 0    & $D$  & 0    \\ \hline
1  & 0   & 0  & 0     & 0    & 0    & 0    & $D$    \\ \hline
\end{tabularx}
\end{table}

По таблице истинности запишем СДНФ функций выходов:
\begin{align*}
D\land \overline{A_1} \land\overline{A_2}\land\overline{A_3} &= Q_0 \\
D\land \overline{A_1} \land\overline{A_2}\land A_3 &= Q_1 \\
D\land \overline{A_1} \land A_2\land\overline{A_3} &= Q_2 \\
D\land \overline{A_1} \land A_2\land A_3 &= Q_3 \\
D\land A_1\land \overline{A_2} \land \overline{A_3} &= Q_4
\end{align*}

Построим в соответствии с полученными функциями комбинационную схему. Схема представлена на рисунке \ref{fig:section3:scheme}:

\begin{figure}[ht!]
    \centering
    \includegraphics[scale=0.08]{S3IM1.png}
    \caption{Комбинационная схема мультиплексора}
    \label{fig:section3:scheme}
\end{figure}

Проведем симуляцию работы получившейся логической схемы в программном обеспечении «\textit{logisim-evoltion}»\cite{logisim}. Результат симуляции представлен на рисунке \ref{fig:section3:simulation}:

\begin{figure}[H]
    \centering
    \includegraphics[scale=0.8]{S3IM2.png}
    \caption{Симуляция работы комбинационной схемы}
    \label{fig:section3:simulation}
\end{figure}

\subsection{Выводы}
В результате выполнения задания по синтезу комбинационный схемы
демультиплексора мы задали правила работы схемы в виде таблицы истинности, синтезировали и проверили её работу в программном обеспечении «\textit{logisim-evoltion}».
